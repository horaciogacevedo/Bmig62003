\documentclass{beamer}
\DeclareFontShape{OT1}{cmss}{b}{n}{<->ssub * cmss/bx/n}{} 
\usetheme{default}
\usepackage{amsmath}
\usepackage{amsfonts}
\usepackage{mathbbol}
\usepackage{xcolor} % before tikz or tkz-euclide if necessary
\usepackage{tkz-euclide} % no need to load TikZ
\usepackage{multirow}
\usepackage{lmodern}
\usepackage{bm}

\usepackage[
backend=biber,
style=authoryear-icomp,
sortlocale=de_DE,
natbib=true,
url=false, 
doi=true,
eprint=false
]{biblatex}
\addbibresource{../../Bibliography/main_ML.bib}



\titlegraphic{\includegraphics[width=2cm]{../../Figures/UAMS_RGB.png}
}


\title{Statistical Machine Learning\\ Multiple Testing}
\author{Horacio G\'omez-Acevedo\\ Department of Biomedical Informatics\\
	University of Arkansas for Medical Sciences}
\begin{document}
	\begin{frame}[plain]
		\maketitle
	\end{frame}
	
	\begin{frame}{Hypothesis test}
		\begin{itemize}
			\item A \textit{statistical hypothesis} $H$ ia a conjecture about the probability distribution of a population.
			\item A hypothesis $H$ is said to be \textit{simple} if the distribution of the population is completely specified by $H$. If not, then $H$ is called a \textit{composite} hypothesis.
			\item $H_0$ is called the \textit{null hypothesis} and $H_1$ the \textit{alternative hypothesis}.
			\item  We say that we commit a  \textit{type I error} if we decide to accept $H_1$, whereas in reality $H_0$ is true.
			\item  The probability of committing a type I error will be denoted by $\alpha$.
			\item Acceptance of $H_0$ whereas $H_1$ is true is called a \textit{type II error}. 
			\item  The probability of committing a type II error will be denoted by $\beta$.  
		\end{itemize}		

	\end{frame}

\begin{frame}{Example}
	A factory has packets of coffee with and adjusted weight of 500 grams. We assume that the weight of the packages is $N(500,50)$-distributed. A container with coffee packages contains packages wrongly processed with weight $N(490,50)$-distributed. To determine the container that has incorrectly weight the packages, we draw a sample (package) of each of two containers $X_1$, $X_2$ . Based on the outcome of the 2-vector $(X_1,X_2)$ we will make a conjecture about the distribution of the population. Namely
	\begin{equation*}
		\begin{split}
			H_0 & \text{: the population is } N(500,50)-\text{distributed}\\
			H_1 & \text{: the population is } N(490,50)-\text{distributed}
		\end{split}
	\end{equation*}
	Let's suppose our decision as\\
	\textit{If both $X_1$ and $X_2 \le 496$ then we accept $H_1$, otherwise we accept $H_0$. }
\end{frame}

\begin{frame}{Example (cont)}
	Let's define our \textit{critical region} $G$ as 
	\begin{equation*}
		G=\{(x_1,x_2)\in \mathbb{R}^2: x_1,x_2 \le 496\}
	\end{equation*}
	Then, we can formulate our decision as
	\begin{equation*}
		=
		\begin{cases}
			\text{if }(X_1,X_2) \in G & \text{then we choose $H_1$ as our conjecture}.\\
			\text{if }(X_1,X_2) \notin G & \text{then we choose $H_0$ as our conjecture}.\\
		\end{cases}	
		\end{equation*}
	We can calculate $\alpha$ and $\beta$ as follows
	\begin{equation*}
		\begin{split}
			\alpha& = P(\text{acceptance of }H_1 | H_0\text{ is true}) \\ 
			& = P((X_1,X_2)\in G| \mu=500, \sigma^2=50) \\
			&=  P(X_1\le 496| \mu=500, \sigma^2=50) \cdot  P(X_2\le 496| \mu=500, \sigma^2=50)\\
			&= 0.081
		\end{split}
	\end{equation*}
\end{frame}

\begin{frame}{Example (cont)}
	\begin{equation*}
		\begin{split}
			\beta& = P(\text{acceptance of }H_0 | H_1\text{ is true}) \\ 
			& = P((X_1,X_2)\notin G| \mu=490, \sigma^2=50) \\
			&= 1-P((X_1,X_2)\in G| \mu=490, \sigma^2=50) \\
			&=  1-P(X_1\le 496| \mu=490, \sigma^2=50) \cdot  P(X_2\le 496| \mu=490, \sigma^2=50)\\
			&= 0.356
		\end{split}
	\end{equation*}
\end{frame}

\begin{frame}{Hypothesis test}
	A \textit{Hypothesis test} is a collection
	\begin{equation*}
		(X_1,\ldots, X_n; H_0; H_1: G)
	\end{equation*}
 where $X_1,\ldots, X_n$ is a sample, $H_0$ and $H_1$ hypotheses concerning the probability distribution of the population and $G\subset \mathbb{R}^n$ a Borel set (meaning a collection of open sets)
 
 If $H_0$ is a simple statistical hypothesis. The \textit{level of significance} of the hypothesis test $(X_1,\ldots, X_n; H_0;H_1;G)$ is understood to be the number
 \begin{equation*}
 	\alpha = P_{X_1,\ldots, X_n}^{H_0}(G)
 \end{equation*}
Thus, we say that $\alpha$ represents the probability of committing a type I error.
\end{frame}

\begin{frame}{The power function}
	With our previous setup the $\beta$ could not be used for composite hypothesis. Thus, we have to define something more general. 
	\begin{itemize}
		\item Let $f(\cdot, \theta))_{\theta \in \Theta}$ be a family of probability densities 
		\item Let's assume that the population $X_1,\ldots, X_n$  has a probability density $f(\cdot,\theta)$ where $\theta \in \Theta$.  
		\item Let's assume that $H_0$ and $H_1$ are statements of the type
		\begin{equation*}
			H_0 \colon \theta \in \Theta_0 \quad \text{and} \quad H_1\colon \theta \in \Theta_1
		\end{equation*}
	where $\Theta_0 \cup \Theta_1= \Theta$ and $\Theta_1 \cap \Theta_1= \emptyset$.
	
	\end{itemize}
	For a fixed $\theta \in \Theta$, the probability distribution of the population is completely specified. Then, we define for every $\theta \in \Theta_1$
	\begin{equation*}
		\beta(\theta)= P_{X_1,\ldots, X_n}^\theta (G^c)
	\end{equation*}
	The expression $1-\beta(\theta)$ is called the \textit{power function} for $\theta \in \Theta_1$.
\end{frame}

\begin{frame}{Example (cont)}
	From our previous example, given is a $N(\mu, 50)$-distributed population, where $\mu \le 500$. 
	The family of probability densities $f(\colon, \mu)$ where $\mu\le 500$ and 
	\begin{equation*}
		f(x,\mu)= \frac{1}{\sqrt{100 \pi}} \exp(\frac{-(x-\mu)^2}{100})
	\end{equation*}
	The parameter space is defined by $\Theta=(-\infty, 500]$. \\
	If we draw a sample $X_1$, $X_2$ of size 2 from this population. We can define $\Theta_0=\{500\}$ and $\Theta_1= (-\infty, 500)$. This corresponds to the following hypotheses
	\begin{equation*}
		H_0 \colon \mu = 500 \quad \text{against} \quad H_1 \colon \mu< 500
	\end{equation*}
If we choose the following critical region 
\begin{equation*}
	G= \{ (x_1,x_2)\in \mathbb{R}^2: x_1, x_2 \le 494.63 \}
\end{equation*}

	
\end{frame}

\begin{frame}{Example (cont)}
	Then, our 5-tuple $(X_1,X_2;H_0;H_1;G)$ constitutes a hypothesis test. 
	
	The size $\alpha$ of the critical region $G$ is $\alpha=0.05$, and the power function
	\begin{equation*}
		\begin{split}
			1-\beta(\mu)&= 1-P^\mu_{X_1,X_2}(G^c)= P^\mu_{X_1,X_2}(G)= P((X_1,X_2)\in G)\\
			&=P(X_1 \le 494.63 \text{ and } X_2 \le 494.63| \mu=\mu, \sigma^2=50) \\
			&=P(X_1 \le 494.63 | \mu=\mu, \sigma^2=50)\\
			&\quad  \cdot P(X_2\le 494.63 | \mu=\mu, \sigma^2=50) 
		\end{split}
	\end{equation*}
\end{frame}

\begin{frame}{Multiple Testing}
	When we are facing testing hypotheses of possibly thousands of significance tests. 
\end{frame}

\begin{frame}{References}
	Materials and some of the pictures are from \citep{pestman}.
	\printbibliography 	
	
	I have used some of the graphs by hacking TiKz code from StakExchange, Inkscape for more aesthetic plots and other old tricks of \TeX
	
\end{frame}


\end{document}
