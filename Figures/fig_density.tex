\documentclass[border=2cm]{standalone}
\usepackage{pgfplots}
\pgfplotsset{compat=1.8}
\begin{document}
	
	\pgfmathdeclarefunction{gauss}{3}{%
		\pgfmathparse{1/(#3*sqrt(2*pi))*exp(-((#1-#2)^2)/(2*#3^2))}%
	}
	
	\begin{tikzpicture}
		\begin{axis}[
			no markers, 
			domain=0:6, 
			samples=100,
			ymin=0,
			axis lines*=left, 
			every axis y label/.style={at=(current axis.above origin),anchor=south},
			every axis x label/.style={at=(current axis.right of origin),anchor=west},
			height=5cm, 
			width=12cm,
			xtick=\empty, 
			ytick=\empty,
			enlargelimits=false, 
			clip=false, 
			axis on top,
			grid = major,
			hide y axis
			]
			
			\begin{scope}[yshift=-\pgflinewidth]
				\clip (axis cs:1,0) rectangle (axis cs:2,0.24);
				\addplot [draw=none,fill=cyan!50] {gauss(x, 3, 1)};
			\end{scope}
			
			\addplot [very thick,blue!50!black] {gauss(x, 3, 1)};
			\pgfmathsetmacro\valueB{gauss(1,1,7)}
			\pgfmathsetmacro\valueA{gauss(1,1,1.65)}
			\pgfmathsetmacro\valueC{gauss(1,1,1)}
			
			\draw [very thick, red]  (axis cs:1,0) -- (axis cs:1,\valueB);
			\draw [very thick, red]  (axis cs:2,0) -- (axis cs:2,\valueA);
			\draw [very thick, red]  (axis cs:3,0) -- (axis cs:3,\valueC);
			
			\node[below] at (axis cs:3,0)  {$\mu=0$}; 
			\node[below] at (axis cs:2,0) {$x=b$};
			\node[below] at (axis cs:1,0) {$x=a$};
			\node at (axis cs:1.3,0.25) {$P(a\le x \le b)$};
		\end{axis}
	\end{tikzpicture}
\end{document}